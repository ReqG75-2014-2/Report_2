Durante a modelagem TO-BE no processo de inscrição, os seguintes indicadores para as metas definidas foram estabelecidos:
\begin{itemize}
	\item{\textbf{Indicador 01}: Produtividade do analisador de dados
		\begin{itemize}
			\item{\textbf{Objeto Medido}: Número de solicitações analisadas pelo analista;}
			\item{\textbf{Responsável}: Gerente CHAMEX;}
			\item{\textbf{Frequência de Medição}: Semestral;}
			\item{\textbf{Local de Medição}: CHAMEX;}
			\item{\textbf{Motivo da Medição}: Identificar a média de solicitações analisadas;}
			\item{\textbf{Modo de Medir}: Número de solicitações / hora trabalhada;}
			\item{\textbf{Meta}: Aumentar a produtividade em 20\%.}
		\end{itemize}}
	\item{\textbf{Indicador 02}: Eficiência Operacional do Processo de Inscrição
		\begin{itemize}
			\item{\textbf{Objeto Medido}: Processo de Inscrição;}
			\item{\textbf{Responsável}: Gerente CHAMEX;}
			\item{\textbf{Frequência de Medição}: Semestral;}
			\item{\textbf{Local de Medição}: CHAMEX;}
			\item{\textbf{Motivo da Medição}: Garantir a qualidade do processo;}
			\item{\textbf{Modo de Medir}: Somatório do tempo das atividades do processo / quantidade de empresas atendidas;}
			\item{\textbf{Meta}: Diminuir pela metade o tempo médio do processo.}
		\end{itemize}}
	\item{\textbf{Indicador 03}: Prazo de Atendimento da Solicitação de Inscrição
		\begin{itemize}
			\item{\textbf{Objeto Medido}: O tempo médio de antedimento;}
			\item{\textbf{Responsável}: Gerente CHAMEX;}
			\item{\textbf{Frequência de Medição}: Semestral;}
			\item{\textbf{Local de Medição}: CHAMEX;}
			\item{\textbf{Motivo da Medição}: Identificar o tempo médio de tramitação do processo em execução;}
			\item{\textbf{Modo de Medir}: Somatório do tempo de envio da solicitação de inscrição e das atividades subsequente até a resposta da solicitação / quantidade de solicitações;}
			\item{\textbf{Meta}: Diminuir o prazo médio de atendimento de solicitação em 20\%.}
		\end{itemize}}
\end{itemize}