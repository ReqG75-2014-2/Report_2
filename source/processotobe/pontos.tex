Levando em conta que a solicitação de participação do MOA era preenchida pela empresa antigamente via planilha Excel e conferida manualmente pelo analisador do MOA, o grupo pensou em qual processo poderia ser atacado de modo a informatizar esses passos.
\\ \indent Embora o processo tenha sofrido alterações como um todo, destacam-se o formulário \emph{web} criado para servir como solicitação de participação no MOA e o \emph{check-list} a ser utilizado pelo analisador para validar ou não a participação da empresa.
\\ \indent Os dados a serem preenchidos no formulário foram todos inseridos nesse contexto da automação, com o objetivo de também conter regras de negócio que antigamente não estavam disponíveis para verificação em tempo real na planilha Excel. Essa melhoria por si só já antecede para o preenchimento do formulário trabalhos realizados anteriormente de forma mais complexa e manual em atividades posteriores, como um campo obrigatório em branco ou violando alguma regra básica de negócio.
\\ \indent Enquanto antigamente a análise da solicitação era feita manualmente e informada à empresa pelo analisador sem o auxílio de uma ferramenta que automatizasse essa atividade, agora, com a melhoria no processo, foi criado um \emph{check-list} em que, caso haja alguma discordância com as regras de negócio identificada pelo analisador, automaticamente um e-mail é enviado à empresa que deseja participar, onde, através da consulta do status da sua solicitação, é possível ela optar em corrigir esses dados, já recuperando respostas no \emph{check-list}. Desse modo o processo de correção tornou-se menos complexo por já trazer dados anteriores preenchidos pela empresa que podem ser aproveitados, além de já apresentar ao responsável pelo preenchimento quais erros foram encontrados no \emph{check-list} realizado. 
\\ \indent A tomada de decisão quanto à corrigir ou não erros no formulário também foi identificada como um ponto de automatização, visto que rapidamente a empresa terá acesso ao formulário a ser corrigido ou abandonará o processo de inscrição.
\\ \indent Os passos descritos anteriormente passaram por um processo de automação, onde através da ferramenta Bizagi Studio o formulário, o check list e a tomada de decisão quanto à correção dos dados foi criada em uma aplicação, sendo possível visualizar através de uma solução derivada diretamente do processo modelado como este novo processo iria funcionar em um contexto real.