A simulação para ambas as versões do TO-BE apresentaram valores consistentes para o tempo esperado para cada atividade no cenário 1. Para o cenário 2 os valores se apresentaram muito superiores para a versão 1, porém a versão 5 não se comportou da mesma maneira para suas atividade, evidenciando assim uma clara melhoria quanto à eficiência da modelagem proposta. 
\\ \indent Evidenciou-se um maior equilíbrio de utilização de recursos na versão 5 do que na 1 do TO-BE. Tal fato também se mostra verdadeiro quando equiparado à utilização de recursos do AS-IS. Em relação ao AS-IS, o avaliador teve um aumento médio de 584,84\%, a empresa teve um aumento de 53,19\%, e o gerente aumentou 114,71\% na média.
\\ \indent O processo TO-BE versão 5 apresentou uma média de horas de execução de 3,16 horas a mais para o cenário 1, e de 10,08 horas a mais para o 2. Este aumento se deu principalmente pela adicão tarefa do gerente “Atribuir solicitação para avaliação”. Devido a essa diferença de tempo não ser muito significativa e também o fato do processo ser executado com mais de 200 horas a menos do que o AS-IS, a versão 5 se mostrou a melhor para ser utilizada no processo de automação.