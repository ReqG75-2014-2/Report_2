O grupo da disciplina de Requisitos de \emph{Software} (RS) ficou responsável por trabalhar, juntamente com o grupo da disciplina de Modelagem de Processos (MPR), soluções de \emph{software} para o contexto da empresa fictícia CHAMEX.
\\ \indent O objetivo da CHAMEX consiste em auxiliar pequenas e médias empresas privadas a melhorarem a qualidade de vida dos trabalhadores. Quanto mais disposição, vitalidade e alegria obtiver entre os trabalhadores, mais resultado positivo as empresas possuem.
\\ \indent Para concretização do suporte às empresas, a CHAMEX elaborou o Modelo de Avaliação (MOA). O principal objetivo desse modelo está atrelado à verificação do nível de satisfação e qualidade de vida dos funcionários de uma determinada organização. A CHAMEX apresenta um modelo de gestão por processos, tendo em vista que não há divisões de departamentos e caracterização hierárquica interna.
\\ \indent O grupo de MPR realizou um levantamento dos processos existentes dentro da CHAMEX e foram identificados:
\begin{itemize}
	\item{Inscrição no MOA;}
	\item{Seleção dos Avaliadores;}
	\item{Avaliação das Empresas;}
	\item{Validação dos Questionários;}
	\item{Compilação dos Resultados.}
\end{itemize}
\ \indent Dessa maneira, foi necessário avaliar qual dos processos descritos anteriormente seria adotado para melhoria. Assim, o grupo de MPR adotou os seguintes critérios:
\begin{itemize}
	\item{Grau de vinculação com os objetivos organizacionais ou com o direcionamento estratégico da organização;}
	\item{Impacto no cliente externo;}
	\item{Potencial para obtenção de benefícios financeiros ou redução de custos para organização;}
	\item{Impacto na imagem externa.}
\end{itemize}
\ \indent Adicionalmente, foi necessário levar em consideração a viabilidade de melhoria de cada processo. Após consideração destes fatores, o grupo de MPR chegou a conclusão de que o processo de Inscrição no MOA seria o mais apropriado para inserção de melhorias, uma vez que os outros processos apresentaram um valor de viabilidade elevado, caracterizando uma implementação complexa. O processo de Inscrição no MOA apresentou um peso significativo e um valor de viabilidade razoável.