O 5W2H representa um conjunto de perguntas sobre um determinado processo ou atividade que procuram explicar com o máximo de clareza possível o entendimento dos colaboradores da empresa quanto ao assunto. Através das respostas extraídas nessa técnica, é possível adquirir o conhecimento necessário para criar um plano de ação que irá promover a mudança \cite{paim}.
\\ \indent O 5W representa as perguntas \emph{What?}, o que será feito; \emph{Who?}, quem irá fazer; \emph{Where?}, onde será feito; \emph{When?}, quando será feito; \emph{Why?}, porque será feito, tentando responder qual a importância daquilo para a empresa.
\\ \indent O 2H representa as perguntas \emph{How?}, como será feito; \emph{How Much?}, qual o custo relativo.

\begin{table}[H]
	\centering
	\begin{tabular}{|c|p{10cm}|}
		\hline
		\textbf{\emph{What?} O que?} & Inscrição MOA \\ \hline
		\textbf{\emph{Who?} Quem?} & CHAMEX e empresa interessada \\ \hline
		\textbf{\emph{Where?} Onde?} & Empresa CHAMEX e site CHAMEX \\ \hline
		\textbf{\emph{When?} Quando?} & Inicio do período de inscrição \\ \hline
		\textbf{\emph{Why?} Por quê?} & Passo inicial necessário para saber quais empresas vão participar do MOA \\ \hline
		\textbf{\emph{How?} Como?} & Empresas se inscrevem no MOA a partir de uma planilha disponibilizada no site da CHAMEX \\ \hline
		\textbf{\emph{How Much?} Quanto?} & - \\ \hline
	\end{tabular}
	\label{tab:5w2h}
	\caption[5W2H no Contexto de Inscrição no MOA]{5W2H no Contexto de Inscrição no MOA.}
\end{table}